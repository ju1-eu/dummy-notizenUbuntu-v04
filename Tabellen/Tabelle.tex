% Integer-Datentypen
\begin{table}[!ht]% hier: !ht 
	\centering
	\caption{Integer-Datentypen}
	\label{tab:Integer-Datentypen}
	%\resizebox{\textwidth}{!}{%
		\begin{tabular}{@{}lrr@{}}
			\toprule
			\multicolumn{1}{c}{\textbf{Typ}} & \multicolumn{1}{c}{\textbf{Wertebereich}} & \multicolumn{1}{c}{\textbf{Speicherbedarf}} \\ \midrule
			int                              & –32768...32767                            & 2 Bytes                                     \\
			int                              & –2147483648...2147483647                  & 4 Bytes                                     \\
			short int                        & –32768...32767                            & 2 Bytes                                     \\
			unsigned short int               & 0...65535                                 & 2 Bytes                                     \\
			long int                         & -2147483648...2147483647                  & 4 Bytes                                     \\
			unsigned long int                & 0...4294967295                            & 4 Bytes                                     \\ \bottomrule
		\end{tabular}%
	%}
\end{table}